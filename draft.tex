\documentclass{ifacconf}

\usepackage{xcolor}
\usepackage{enumerate}
\usepackage{amsmath, amssymb}
\usepackage{natbib}            % you should have natbib.sty
\usepackage{graphicx}          % Include this line if your 
                               % document contains figures,
%\usepackage[dvips]{epsfig}    % or this line, depending on which
                               % you prefer.
% predefined environments
%\begin{thm} ... \end{thm}		% Theorem
%\begin{lem} ... \end{lem}		% Lemma
%\begin{claim} ... \end{claim}	% Claim
%\begin{conj} ... \end{conj}	% Conjecture
%\begin{cor} ... \end{cor}		% Corollary
%\begin{fact} ... \end{fact}	% Fact
%\begin{hypo} ... \end{hypo}	% Hypothesis
%\begin{prop} ... \end{prop}	% Proposition
%\begin{crit} ... \end{crit}	% Criterion
%\begin{rem} ... \end{rem}    % Remark
%\begin{pf} ... \end{pf}      % Proof
%\begin{ack} ... end{ack}     % Acknowledgement

\providecommand{\abs}[1]{\left|#1\right|}
\providecommand{\norm}[1]{\left\|#1\right\|}
\newcommand{\blue}{\textcolor{blue}}
\newcommand{\green}[1]{\textcolor[rgb]{0,.5,0}{#1}}
\newcommand{\red}{\textcolor{red}}

\allowdisplaybreaks[3]

\begin{document}

\begin{frontmatter}

\title{Approximating the Maximal Positive Invariant Set of Parameter Uncertain Linear System}%\thanksref{footnoteinfo}} 
% Title, preferably not more than 10 words.


\author{Rainer M. Schaich} 
\author{Mark Cannon}
\author{Sina Ober-Bl\"obaum}


\address{Department of Engineering Science, University of Oxford, OX1 3PJ, UK. (e-mail: \{rainer.schaich,mark.cannon,sina.ober-blobaum\}@eng.ox.ac.uk).}


          
\begin{keyword}                           % Five to ten keywords,  
Keyword 1; keyword 2;
\end{keyword}                             % keyword list or with the 
                                          % help of the Automatica 
                                          % keyword wizard


\begin{abstract}                          % Abstract of not more than 250 words.
Abstract goes here.
\end{abstract}

\end{frontmatter}


\section{Introduction}
%
%
\section{Set Operations for Parametrically Convex Sets}
%
%
\section{Multiplicative to Additive Approximation}

\section{Computational Examples}

\section{Conclusion}
\end{document}